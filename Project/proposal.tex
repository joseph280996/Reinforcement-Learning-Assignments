\documentclass{article}
\usepackage[utf8]{inputenc}

\title{Reinforcement learning for raceline prediction}
\author{Joseph Hadidjojo, Tung Pham}
\date{\vspace{-1em}}

\usepackage{natbib}
\usepackage{graphicx}

\begin{document}

\maketitle

% for final report only
%\begin{abstract}
%A one paragraph high level description of the work. The abstract is typically the first thing someone would read from the paper before deciding whether to continue reading and hence, serves as an advertisement to the reader to read the whole paper. 
%\end{abstract}

\section{Introduction}

The Formula 1 is a open-wheel single-seater formula racing cars. Its tournament,
the FIA Formula One World Championship, has been one of the world's premier forms of racing since 1950.
Participants are all top racers, however, the champions are those that talented
in choosing the correct racing line for a given track. This put them
leagues above the others. However, most of these are estimates based on the
experience of racers and it would be useful for them to be able to see the ideal
racing line during practice and training.

In this project, we're aiming to solve this problem with the help of
Reinforcement Learning techniques in order to help visualize the mistake that
racers made to allow better adjustment and practice session. Specifically, we'll
train an agent that can identify the optimal racing line given a track using
multiple on-policy and off-policy update methods. That way, we can highlight the
mistake that the racer made in comparison with the optimal racing line.

\section{Background Related Work}

Following, you should provide the necessary background and discuss related work in the RL literature. This section should also be about a page. Citations should be in BibTeX format \citep{thrun2005probabilistic}. Some approaches have proposed using RL for robot soccer \cite{riedmiller2009reinforcement}.

\section{Technical Approach}
\subsection{Environment Setup}
To simulate the racing environment, we'll assume that the environment has an
idealistic design where there are no elevation or bumps on the track. We'll also
ignore all types of friction like air friction and terrain. 
The width of racetrack is kept fixed throughout and the agent can only choose
the actions that select a fixed value for acceleration and deceleration of the
car at each time step. There is,
however, a cap speed which limits the velocity.
All elements regarding the racecar like tire degradation, inertia, momentum,
etc. are also ignored. 

In the scenario that the agent failed to take a curve and is thrown off the
racetrack, the agent will continue from the nearest location on the racetrack in
the next timestep and the velocity as well as the acceleration is resetted to
zero. This will required the agents to again increase the acceleration and
velocity 

At first, we'll be using the racing car
environment from gymnasium Python package for simplicity. Then, to add more
complications into the environment later on, we'll move toward a more flexible
environment that is more tailored to our specific need.
\subsection{Methodology}
\subsubsection{Baseline}
\subsubsection{Deep Q-Network}
\subsubsection{Deep Deterministic Q-Network}
\subsubsection{Proximal Policy Optimization}
\section{Evaluation}
\section{Timeline and Individual Responsibilities}

\bibliographystyle{plain}
\bibliography{references}
\end{document}

